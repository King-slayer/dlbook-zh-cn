\chapter{数学符号}
\label{ch:notation}

这一部分提供了一个简明的索引,描述整本书中使用的数学符号。如果你不熟悉任何对应的
数学概念,这个符号索引可能看起来很吓人。然而,不要绝望,我们在 2 --- 4 章里描述了
这些概念的大部分。

\begin{center}
  \setstretch{1.5}
  {\Large\bfseries 数和数组}\\
  \vspace{1em}
  \begin{tabular}{c l}
    $a$ & 标量(整数或实数)\\
    $\pmb{a}$ & 向量 \\
    $\pmb{A}$ & 矩阵 \\
    $\pmb{\mathsf{A}}$ & 张量 \\
    $\pmb{I}_n$ & $n$ 行 $n$ 列的单位矩阵 \\
    $\pmb{I}$ & 单位矩阵,其维度隐含在上下文中 \\
    $\pmb{e}^{(i)}$ & 标准的基向量 $[0, \ldots, 0, 1, 0, \ldots, 0]$,$1$ 的位置由 $i$ 确定 \\
    $\mathrm{diag}(\pmb{a})$ & 对角线元素为 $\pmb{a}$ 的对角矩阵 \\
    $\mathrm{a}$ & 标量随机变量 \\
    $\mathbf{a}$ & 向量值随机变量 \\
    $\mathbf{A}$ & 矩阵值随机变量
  \end{tabular}
\end{center}

\vspace{1em}

\begin{center}
  \setstretch{1.5}
  {\Large\bfseries 集合和图}\\
  \vspace{1em}
  \begin{tabular}{c l}
    $\mathbb{A}$ & 集合 \\
    $\mathbb{R}$ & 实数集合 \\
    $\{0,1\}$ & 包含 $0$ 和 $1$ 的集合 \\
    $\{0,1,\ldots,n\}$ & $0$ 到 $n$ 的所有整数集合 \\
    $[a,b]$ & 包括 $a$ 和 $b$ 的实数区间 \\
    $(a,b]$ & 左开右闭区间,不包括 $a$ 但包括 $b$ \\
    $\mathbb{A}\backslash\mathbb{B}$ & 集合差,例如,包含有 $\mathbb{A}$ 的元素但其不在 $\mathbb{B}$ 中的集合 \\
    $\mathcal{G}$ & 图 \\
    $Pa_{\mathcal{G}}(x_i)$ & 图  $\mathcal{G}$ 中 $x_i$ 的父顶点
  \end{tabular}
\end{center}

\vspace{1em}

\begin{center}
  \setstretch{1.5}
  {\Large\bfseries 索引}\\
  \vspace{1em}
  \begin{tabular}{c l}
    $a_i$ & 向量 $\pmb{a}$ 的第 $i$ 个元素,索引起始位置为 $1$ \\
    $a_{-i}$ & 向量 $\pmb{a}$ 中除了第 $i$ 个元素之外的所有其它元素 \\
    $A_{i,j}$ & 矩阵 $\pmb{A}$ 的 $i,j$ 位置元素 \\
    $A_{i,:}$ & 矩阵 $\pmb{A}$ 的第 $i$ 行 \\
    $A_{:,i}$ & 矩阵 $\pmb{A}$ 的第 $i$ 列 \\
    $\mathsfit{A}_{i,j,k}$ & 三维张量 $\pmb{\mathsf{A}}$ 的 $(i,j,k)$ 位置的元素 \\
    $\pmb{\mathsf{A}}_{:,:,i}$ & 三维张量在 $i$ 处的二维切分 \\
    $\mathrm{a}_i$ & 随机向量 $\mathbf{a}$ 的第 $i$ 个元素
  \end{tabular}
\end{center}

\vspace{1em}

\begin{center}
  \setstretch{1.5}
  {\Large\bfseries 线性代数操作}\\
  \vspace{1em}
  \begin{tabular}{c l}
    $\pmb{A}^{\top}$ & 矩阵 $\pmb{A}$ 的转置 \\
    $\pmb{A}^+$ & 矩阵 $\pmb{A}$ 的摩尔--彭若斯广义逆 \\
    $\pmb{A} \odot \pmb{B}$ & 矩阵 $\pmb{A}$ 和 $\pmb{B}$ 的按元素(阿达玛)乘积 \\
    $\mathrm{det}(\pmb{A})$ & 矩阵 $\pmb{A}$ 的行列式
  \end{tabular}
\end{center}

\vspace{1em}

\begin{center}
  \setstretch{1.5}
  {\Large\bfseries 微积分}\\
  \vspace{1em}
  \begin{tabular}{c l}
    $\frac{dy}{dx}$ & $y$ 对 $x$ 求导 \\
    $\frac{\partial y}{\partial x}$ & $y$ 对 $x$ 求偏微分 \\
    $\nabla_{\pmb{x}}y$ & $y$ 在 $\pmb{x}$ 上的梯度 \\
    $\nabla_{\pmb{X}}y$ & $y$ 对 $\pmb{X}$ 求矩阵导数 \\
    $\nabla_{\pmb{\mathsf{X}}}y$ & 包含 $y$ 对 $\pmb{\mathsf{X}}$ 求得的导数的张量 \\
    $\frac{\partial f}{\partial \pmb{x}}$ & $f : \mathbb{R}^n \rightarrow \mathbb{R}^m$ 的雅可比矩阵 $\pmb{J} \in \mathbb{R}^{m \times n}$ \\
    $\nabla^2_{\pmb{x}}f(\pmb{x})$ 或 $\pmb{H}(f)(\pmb{x})$ & $f$ 在输入点 $\pmb{x}$ 的海森矩阵\\
    $\displaystyle\int f(\pmb{x})d\pmb{x}$ & 在整个 $\pmb{x}$ 上的定积分\\
    $\displaystyle\int_{\mathbb{S}} f(\pmb{x})d\pmb{x}$ & 在集合 $\mathbb{S}$ 上对 $\pmb{x}$ 的定积分 \\
  \end{tabular}
\end{center}

\vspace{1em}

\begin{center}
  \setstretch{1.5}
  {\Large\bfseries 概率与信息论}\\
  \vspace{1em}
  \begin{tabular}{c l}
    $\mathrm{a} \bot \mathrm{b}$ & 随机变量 $\mathrm{a}$ 与 $\mathrm{b}$ 互相独立 \\
    $\mathrm{a} \bot \mathrm{b}\: | \: \mathrm{c}$ & 给定 $\mathrm{c}$ 时它们条件独立 \\
    $P(\mathrm{a})$ & 一个离散变量上的概率分布 \\
    $p(\mathrm{a})$ & 一个连续变量,或者类型没有指定的变量上的概率分布 \\
    $\mathrm{a} \sim P$ & 随机变量 $\mathrm{a}$ 具有 $P$ 的分布 \\
    $\mathbb{E}_{\mathrm{x} \sim P} [f(x)]$ 或 $\mathbb{E}f(x)$ & $f(x)$ 对 $P(\mathrm{x})$ 的期望 \\
    $\mathrm{Var}(f(x))$ & 在 $P(\mathrm{x})$ 下 $f(x)$ 的方差 \\
    $\mathrm{Cov}(f(x),g(x))$ & 在 $P(\mathrm{x})$ 下 $f(x)$ 和 $g(x)$ 的协方差 \\
    $H(\mathrm{x})$ & 随机变量 $\mathrm{x}$ 的香农熵 \\
    $D_{\mathrm{KL}}(P \parallel Q)$ & $\mathrm{P}$ 和 $\mathrm{Q}$ 的 $\mathrm{KL}$ 散度 \\
    $\mathcal{N}(\pmb{x};\pmb{\mu};\pmb{\Sigma})$ & $\pmb{x}$ 上均值为 $\pmb{\mu}$,协方差为 $\pmb{\Sigma}$ 的高斯分布 \\
  \end{tabular}
\end{center}

\vspace{1em}

\begin{center}
  \setstretch{1.5}
  {\Large\bfseries 函数}\\
  \vspace{1em}
  \begin{tabular}{c l}
    $f : \mathbb{A} \rightarrow \mathbb{B}$ & 定义域为 $\mathbb{A}$,值域为 $\mathbb{B}$ 的函数 $f$ \\
    $f \circ g$ & 函数 $f$ 和 $g$ 的组合 \\ % 复合函数
    $f(\pmb{x};\pmb{\theta})$ & $\pmb{x}$ 的函数,$\pmb{x}$ 由 $\pmb{\theta}$ 参数化。有时候我们只写 $f(\pmb{x})$,忽略参数 $\pmb{\theta}$ 来简化符号。 \\
    $\mathrm{log}\; x$ & $x$ 的自然对数 \\
    $\sigma(x)$ & {\serif Logistic sigmoid},$\frac{1}{1 + \mathrm{exp}(-x)}$ \\
    $\zeta(x)$ & {\serif Softplus},$\mathrm{log}(1 + \mathrm{exp}(x))$ \\
    $||\pmb{x}||_p$ & $\pmb{x}$ 的 $L^p$ 模 \\
    $||\pmb{x}||$ & $\pmb{x}$ 的 $L^2$ 模 \\
    $x^+$ & $x$ 的正数部分,即 $\mathrm{max}(0,x)$ \\
    $\pmb{1}_{\mathrm{condition}}$ & 如果条件真为 $1$,否则为 $0$ \\
  \end{tabular}
\end{center}

有时候我们使用一个参数是一个标量的函数 $f$,但是将它应用到一个向量,矩阵,
或者张量:$f(\pmb{x})$,$f(\pmb{X})$,或 $f(\pmb{\mathsf{X}})$。
这意思是将 $f$ 按元素应用到数组。例如,如果 $\pmb{\mathsf{C}} = \sigma(\pmb{\mathsf{X}})$,
那么对于所有有效的 $i$,$j$ 和 $k$ 的值,$\mathsfit{C}_{i,j,k} = \sigma(\mathsfit{X}_{i,j,k})$。

\vspace{1em}

\begin{center}
  \setstretch{1.5}
  {\Large\bfseries 数据集和分布}\\
  \vspace{1em}
  \begin{tabular}{c l}
    $p_{\mathrm{data}}$ & 数据生成的分布 \\
    $\hat{p}_{\mathrm{data}}$ & 由训练集定义的经验分布 \\
    $\mathbb{X}$ & 一个训练样本集 \\
    $\pmb{x}^{(i)}$ & (输入自)数据集的第 $i$ 个样本 \\
    $y^{(i)}$ 或 $\pmb{y}^{(i)}$ & 与 $\pmb{x}^{(i)}$ 关联的有监督学习的目标 \\
    $\pmb{X}$ & $m \times n$ 矩阵,其具有在 $\pmb{X}_i$ 行中的输入样本 $\pmb{x}^{(i)}$ \\
    % TODO: 原文最后有个冒号,可能还没完成
  \end{tabular}
\end{center}
