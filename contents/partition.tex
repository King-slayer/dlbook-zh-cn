\chapter{处理配分函数}
\label{ch:partition}

在 16.2.2 节,我们看到了很多概率模型(通常是无向图模型)是由一个没有正规化的概率分布 $\tilde{p}(\mathbf{x};\theta)$ 定义的. 我们必须通过除以一个配分函数 $Z(\pmb{\theta})$ 来正规化 $\tilde{p}$ 才能得到一个合法的概率分布:

\begin{equation}  \label{eq:pyth}
p(\mathbf{x};\pmb{\theta}) = \frac{1}{Z(\pmb{\theta})} \tilde{p}(\mathbf{x}; \pmb{\theta}).
\end{equation}

配分函数对连续型变量是积分,而对离散型变量是对所有状态的非正规化概率进行求和:

\begin{equation}  \label{eq:pyth}
\int \!\tilde{p}(\pmb{x})\, \mathrm{d}\pmb{x}
\end{equation}

或者

\begin{equation}  \label{eq:pyth}
\sum_{x} \tilde{p}(\pmb{x}).
\end{equation}

这个操作对很多有趣的模型都是难解的. 

如我们将在第 20 章中看到的,有些深度学习模型被设计成有可解的正规化常量,或者被设计成不需要包含计算 $p(\mathbf{x})$ 的. 然而,其他的模型需要直接面对难解的配分函数的挑战. 本章,我们会给出一些用来训练和评估那些有着难解配分函数的模型的技术. 

\section{对数似然梯度 log-likelihood gradient}
\label{sec:llg}

让通过最大似然估计在学习无向的模型中尤其困难的原因是配分函数依赖于参数. 对数似然函数关于参数的梯度包含一个配分函数的梯度项:

\begin{equation}  \label{eq:pyth}
\nabla_{\pmb{\theta}} \log p(\mathbf{x};\pmb{\theta}) = \nabla_{\pmb{\theta}} \log \tilde{p}(\mathbf{x};\pmb{\theta}) - \nabla_{\pmb{\theta}} \log Z\pmb{\theta})
\end{equation}

这是一个著名的分解——将学习分成正负两个部分.\\

对大多数有趣的无向图模型,负部(negative phase)是困难的. 没有隐含变量或者隐含变量间的交互很少的模型通常会有可解的正部(positive phase) 典型的包含一个直接简单的正部而非常困难的负部的模型是 RBM,它的隐含单元在给定可见单元的时候是彼此是条件独立的. 而正部困难的例子是,隐含元之间的交互特别复杂,这个在第 19 章会讲述. 本章聚焦在负部的难解性上.\\

让我们仔细看看 $\log Z$ 的梯度:

\begin{equation}  \label{eq:pyth}
\frac{\partial}{\partial \pmb{\theta}} \log Z
\end{equation}
\begin{equation}  \label{eq:pyth}
=\frac{\frac{\partial}{\partial \pmb{\theta}} Z}{Z}
\end{equation}
\begin{equation}  \label{eq:pyth}
=\frac{\frac{\partial}{\partial \pmb{\theta}} \sum_{\mathbf{x} \tilde{p}(\mathbf{x})}}{Z}
\end{equation}
\begin{equation}  \label{eq:pyth}
=\frac{\sum_{\mathbf{x} \frac{\partial}{\partial \pmb{\theta}} \tilde{p}(\mathbf{x})}}{Z}
\end{equation}

对于那些对所有的 $\mathbf{x}$ 保证 $p(\mathbf{x}) > 0$ 的模型,我们可以为 $\tilde{p}(\mathbf{x})$ 替换 $\exp(\log \tilde{p}(\mathbf{x})$:
\begin{equation}  \label{eq:pyth}
\frac{\sum_{\mathbf{x} \frac{\partial}{\partial \pmb{\theta}} \exp(\log \tilde{p}(\mathbf{x}))}}{Z}
\end{equation}
\begin{equation}  \label{eq:pyth}
=\frac{\sum_{\mathbf{x} \exp(\log\tilde{p}(\mathbf{x})) \frac{\partial}{\partial \pmb{\theta}} \tilde{p}(\mathbf{x})}}{Z}
\end{equation}
\begin{equation}  \label{eq:pyth}
=\frac{\sum_{\mathbf{x} \tilde{p}(\mathbf{x}) \frac{\partial}{\partial \pmb{\theta}} \tilde{p}(\mathbf{x})}}{Z}
\end{equation}
\begin{equation}  \label{eq:pyth}
=\sum_{x}p(\mathbf{x}) \frac{\partial}{\partial \pmb{\theta}} \log \tilde{p}(\mathbf{x})
\end{equation}
\begin{equation}  \label{eq:pyth}
=\mathbf{E}_{x\sim p(\mathbf{x})} \frac{\partial}{\partial \pmb{\theta}} \tilde{p}(\mathbf{x}).
\end{equation}

这个推导对离散的 $\pmb{x}$ 进行求和,类似地也可以对连续的 $\pmb{x}$ 进行积分. 在连续版本的推到中,我们使用 Leibniz 法则在积分符号下进行微分,获得等式:
\begin{equation}  \label{eq:pyth}
\frac{\partial}{\partial \pmb{\theta}} \int \!\tilde{p}(\mathbf{x}) \, \mathrm{d}\pmb{x} = \int \!\frac{\partial}{\partial \pmb{\theta}} \tilde{p}(\pmb{x}) \,\mathrm{d}\pmb{x}.
\end{equation}

等式只有在 $\tilde{p}$ 和 $\frac{\partial}{\partial \pmb{\theta}} \tilde{p}(\mathbf{x})$ 满足某种规范化条件时可用. 用测度论的术语,条件是:
\begin{enumerate*}[label={\roman*)}]
\item $\tilde{p}$ 必须是一个对每个 $\pmb{\theta}$ 值的 $\pmb{x}$ 的 Lebesgue-可积函数;
\item $\frac{\partial}{\partial \pmb{\theta}} \tilde{p}(\mathbf{x})$ 必须对所有的 $\pmb{\theta}$ 和几乎所有的 $\pmb{x}$ 存在
\item 必须存在一个可积分函数 $R(\pmb{x})$ 控制住 $\frac{\partial}{\partial \pmb{\theta}} \tilde{p}(\mathbf{x})$(如对所有的 $\pmb{\theta}$ 和几乎所有的 $\pmb{x}$ 有 $|\frac{\partial}{\partial \pmb{\theta}}\tilde{p}(\mathbf{x})|\leq R(\pmb{x})$).
\end{enumerate*}

幸运的是,大多数机器学习模型满足这些条件.\\

这个等式
\begin{equation}  \label{eq:pyth}
\nabla_{\pmb{\theta}} Z = \mathbb{E}_{\mathbf{x}\sim p(\mathbf{x})} \nabla_{\pmb{\theta}} \log \tilde{p}(\mathbf{x})
\end{equation}
是一系列 Monte Carlo 方法的基础,可以用来近似最大化那些有着难解的配分函数的模型的似然函数.\\

学习无向模型的 Monte Carlo 方法给出了一个直觉框架,我们可以利用这样的框架来思考正部和负部. 在正部中,我们对那些从数据抽取的 $\pmb{x}$ 增加 $\log \tilde{p}(\mathbf{x})$. 在负部中,我们通过减少从模型分布中个采样的 $\log \tilde{p}(\mathbf{x})$ 来降低配分函数.\\

在深度学习文献中,通常会用能量函数参数化 $\log \tilde{p}$(Eq. 16.7). 在这种情况下,我们可以将正部解释为降低训练样本的能量,而负部则是提升从模型中采样的样本的能量,如图 18.1 所示.


