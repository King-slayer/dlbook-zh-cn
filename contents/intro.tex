% intro.tex

\chapter{介绍}
\label{ch:intro}

Inventors have long dreamed of creating machines that think. This desire dates
back to at least the time of ancient Greece. The mythical figures Pygmalion,
Daedalus, and Hephaestus may all be interpreted as legendary inventors, and
Galatea, Talos, and Pandora may all be regarded as artificial life (Ovid and Martin,
2004; Sparkes, 1996; Tandy, 1997).

When programmable computers were first conceived, people wondered whether
they might become intelligent, over a hundred years before one was built (Lovelace,
1842). Today, artificial intelligence (AI) is a thriving field with many practical
applications and active research topics. We look to intelligent software to automate
routine labor, understand speech or images, make diagnoses in medicine and
support basic scientific research.

In the early days of artificial intelligence, the field rapidly tackled and solved
problems that are intellectually difficult for human beings but relatively straight-
forward for computers—problems that can be described by a list of formal, math-
ematical rules. The true challenge to artificial intelligence proved to be solving
the tasks that are easy for people to perform but hard for people to describe
formally—problems that we solve intuitively, that feel automatic, like recognizing
spoken words or faces in images.

\section{谁应该读这本书?}

\section{深度学习的历史趋势}

\subsection{不断增长的模型规模}
\label{subsec:increasing_model_sizes}