%% glossaries

%% Chapter 1

\newglossaryentry{ai}{
  name={人工智能},
  description={\emph{Artifical Intelligence}, AI}
}

\newglossaryentry{dl}{
  name={深度学习},
  description={\emph{Deep Learning}}
}

\newglossaryentry{knowledge-base}{
  name={知识库},
  description={\emph{Knowledge Base}}
}

\newglossaryentry{ml}{
  name={机器学习},
  description={\emph{Machine Learning}}
}

\newglossaryentry{logistic-regression}{
  name={逻辑回归},
  description={\emph{Logistic Regression}}
}

\newglossaryentry{naive-bayes}{
  name={朴素贝叶斯},
  description={\emph{naive Bayes}}
}

\newglossaryentry{representations}{
  name={表征},
  description={\emph{Representations},表征是信息的呈现方式}
}

\newglossaryentry{rep-learning}{
  name={表征学习},
  description={\emph{Representation Learning}}
}

\newglossaryentry{autoencoder}{
  name={自编码器},
  description={\emph{Autoencoder (s)}}
}

\newglossaryentry{encoder}{
  name={编码器},
  description={\emph{encoder}}
}

\newglossaryentry{decoder}{
  name={解码器},
  description={\emph{decoder}}
}

\newglossaryentry{fov}{
  name={变化因素},
  description={\emph{factors of variation}}
}

\newglossaryentry{mlp}{
  name={多层感知器},
  description={\emph{multilayer perceptron}}
}

%% Chapter 2

\newglossaryentry{scalar}{
  name={标量},
  description={scalar}
}

\newglossaryentry{scalars}{
  name={标量},
  description={Scalars}
}

\newglossaryentry{vec}{
  name={向量},
  description={vector}
}

\newglossaryentry{vecs}{
  name={向量},
  description={Vectors}
}

\newglossaryentry{matrix}{
  name={矩阵},
  description={matrix}
}

\newglossaryentry{matrices}{
  name={矩阵},
  description={Matrices}
}

\newglossaryentry{tensor}{
  name={张量},
  description={tensor}
}

\newglossaryentry{tensors}{
  name={张量},
  description={Tensors}
}

\newglossaryentry{transpose}{
  name={转置},
  description={transpose}
}

\newglossaryentry{main-diag}{
  name={主对角线},
  description={main diagonal}
}

\newglossaryentry{broadcasting}{
  name={广播},
  description={broadcasting}
}

\newglossaryentry{matrix-product}{
  name={矩阵积},
  description={matrix product}
}

\newglossaryentry{element-product}{
  name={按元素乘积},
  description={element-wise product}
}

\newglossaryentry{hadamard-product}{
  name={阿达马乘积},
  description={Hadamard product}
}

\newglossaryentry{dot-product}{
  name={点乘},
  description={dot product}
}

\newglossaryentry{matrix-inversion}{
  name={矩阵求逆},
  description={matrix inversion}
}

\newglossaryentry{identity-matrix}{
  name={单位矩阵},
  description={identity matrix}
}

\newglossaryentry{linear-comb}{
  name={线性组合},
  description={linear combination}
}

\newglossaryentry{span}{
  name={生成空间},
  description={span}
}

\newglossaryentry{column-space}{
  name={列空间},
  description={column space}
}

\newglossaryentry{range}{
  name={范围},
  description={range}
}

\newglossaryentry{linear-dep}{
  name={线性相关},
  description={linear dependence}
}

\newglossaryentry{linearly-dep}{
  name={线性无关},
  description={linearly dependent}
}

\newglossaryentry{linearly-indep}{
  name={线性无关},
  description={linearly independent}
}

\newglossaryentry{linear-indep}{
  name={线性无关},
  description={linear independent}
}

\newglossaryentry{square}{
  name={方的},
  description={square}
}

\newglossaryentry{singular}{
  name={奇异矩阵},
  description={singular}
}

%% Chapter 12

\newglossaryentry{weight}{
  name={权重},
  description={\emph{Weight}}
}

\newglossaryentry{bias}{
  name={偏置},
  description={\emph{Bias}}
}

\newglossaryentry{SGD}{
  name=SGD,
  description={\emph{Stochastic Gradient Descent}, 随机梯度下降算法}
}

\newglossaryentry{warps}{
  name={线程束},
  description={\emph{Warps},同时运行的一组线程的称呼}
}

\newglossaryentry{overfitting}{
  name={过度拟合},
  description={\emph{overfitting},过度拟合,过拟合,过适}
}

\newglossaryentry{generalization_error}{
  name={泛化误差},
  description={\emph{Generalization error},泛化误差}
}

\newglossaryentry{dropout}{
  name={弃权},
  description={\emph{Dropout}, 弃权}
}

\newglossaryentry{bdt}{
  name={提高决策树},
  description={\emph{Boosted decision trees}}
}

\newglossaryentry{gcn}{
  name={全局对比度归一化},
  description={\emph{Global contrast normalization}, 全局对比度归一化}
}

\newglossaryentry{minibatch}{
  name={小批量},
  description={\emph{Minibatch}, 小批量}
}
