%% glossaries

%% Chapter 1

\newglossaryentry{ai}{
  name={人工智能},
  description={\emph{Artifical Intelligence}, AI}
}

\newglossaryentry{dl}{
  name={深度学习},
  description={\emph{Deep Learning}}
}

\newglossaryentry{knowledge-base}{
  name={知识库},
  description={\emph{Knowledge Base}}
}

\newglossaryentry{ml}{
  name={机器学习},
  description={\emph{Machine Learning}}
}

\newglossaryentry{logistic-regression}{
  name={逻辑回归},
  description={\emph{Logistic Regression}}
}

\newglossaryentry{representations}{
  name={表征},
  description={\emph{Representations},表征是信息的呈现方式}
}

\newglossaryentry{rep-learning}{
  name={表征学习},
  description={\emph{Representation Learning}}
}

\newglossaryentry{autoencoder}{
  name={自动编码器},
  description={\emph{Autoencoder (s)}}
}

\newglossaryentry{encoder}{
  name={编码器},
  description={\emph{encoder}}
}

\newglossaryentry{decoder}{
  name={解码器},
  description={\emph{decoder}}
}

%% Chapter 12

\newglossaryentry{weight}{
  name={权重},
  description={\emph{Weight}}
}

\newglossaryentry{bias}{
  name={偏置},
  description={\emph{Bias}}
}

\newglossaryentry{SGD}{
  name=SGD,
  description={\emph{Stochastic Gradient Descent}, 随机梯度下降算法}
}

\newglossaryentry{warps}{
  name={线程束},
  description={\emph{Warps},同时运行的一组线程的称呼}
}

\newglossaryentry{overfitting}{
  name={过度拟合},
  description={\emph{overfitting},过度拟合,过拟合,过适}
}

\newglossaryentry{generalization_error}{
  name={泛化误差},
  description={\emph{Generalization error},泛化误差}
}

\newglossaryentry{dropout}{
  name={弃权},
  description={\emph{Dropout}, 弃权}
}

\newglossaryentry{bdt}{
  name={提高决策树},
  description={\emph{Boosted decision trees}}
}

\newglossaryentry{gcn}{
  name={全局对比度归一化},
  description={\emph{Global contrast normalization}, 全局对比度归一化}
}
